 \section{ЗАКЛЮЧЕНИЕ}

В рамках первой части настоящей работы была программно реализована столбчатая модель глобальной электрической цепи (ГЭЦ). Источники в модели задавались путем выделения столбцов воздуха, характеризуемых глубокой конвекцией, по критерию, основанному на значении доступной конвективной потенциальной энергии; площадь области смешанной фазы в данном столбце оценивалась через отношение величины осадков к величине запасенной влаги в столбе воздуха. \textbf{При помощи созданной модели ГЭЦ можно анализировать задачи, связанные с крупномасштабными возмущениями проводимости воздуха.}

% Была произведена серия расчетов ионосферного потенциала (ИП). Каждая серия расчетов производилась с различной параметризацией проводимости атмосферы: с простой экспоненциальной проводимостью и с более сложной параметризацией, учитывающей влияние солнечной активности, при различных уровнях солнечной активности. В результате каждой серии расчетов был получен 24-часовой набор данных ИП для каждого третьего дня 2016 года.

Разработанная модель ГЭЦ позволила получить кривую суточной вариации ионосферного потенциала (ИП), которая близка к экспериментальной кривой Карнеги, что подтверждает корректность работы созданной реализации столбчатой модели ГЭЦ. Анализ результатов расчетов ИП показал, что \textbf{форма кривой суточной вариации ИП, которая получается при моделировании с экспоненциальной проводимостью, совпадает с кривой, которая получается при моделировании с учетом более реалистичной проводимости.} Кроме того, удалось установить, что \textbf{учет более реалистичной проводимости приводит к повышению ИП примерно на 10~кВ в фазу максимума солнечной активности и к понижению примерно на те же 10~кВ в фазу минимума солнечной активности} относительно средних значений ИП.

Во второй части работы было обнаружено, что \textbf{как моделируемый ИП, так и измеряемый на полярной станции Восток приповерхностный градиент потенциала (ГП) электрического поля имеют синусоидальные вариации по фазам колебания Мад\-де\-на--Джу\-ли\-ана (КМД);} такие вариации имеют статистически значимые корреляции с выбираемыми соответствующим образом проекциями двумерного индекса RMM (Real-time Multivariate MJO index), характеризующего КМД.

Более глубокое исследование с использованием эмпирических ортогональных функций, вычисляемых на основе вкладов в ИП, позволило выделить во вкладах паттерны, близкие к тем паттернам конвекции, которые характерны для КМД. \textbf{Большая часть изменчивости ИП на масштабах КМД объясняется двумя базовыми колебаниями конвекции, одно из которых происходит над Юго-Восточной Азией, а другое --- над Индийским океаном.} Такие колебания имеют естественную связь с компонентами индекса RMM. Кроме того, на основе вкладов в ИП, которые имеют ясный физический смысл, можно даже выделить новый индекс КМД.

Несмотря на небольшое расхождение по фазе между вариациями \textbf{ИП и ГП} на масштабах КМД (вероятно, связанное с недостатками моделирования ИП и с влиянием локальных эффектов на результаты измерений ГП), обе величины \textbf{показывают высокую и статистически значимую корреляцию с циклом КМД;} вместе с более ранними результатами о связи ЭНЮК и ГЭЦ \cite{Slyunyaev_et_al_2021a, Slyunyaev_et_al_2021b, Harrison_et_al_2011, Slyunyaev_et_al_2021c, Lavigne_et_al_2017} это позволяет говорить об отражении различных климатических мод в атмосферном электричестве.

%Было обнаружено расхождение результатов моделирования ИП с результатами измерений ГП: существует фазовый сдвиг между вариациями ИП и ГП на масштабах КМД. Данный сдвиг составляет примерно полторы фазы КМД и может быть списан на ошибки моделирования ИП и на влияние локальных эффектов на измерения ГП. Однако точной причины наблюдаемого расхождения пока не установлено. Первым шагом на пути к пониманию причин наличия расхождения может стать долгосрочное моделирование ГЭЦ с более реалистичной проводимостью, учитывающей влияние солнечной активности. Для этих целей может быть использована столбчатая модель ГЭЦ, разработанная в первой части работы.