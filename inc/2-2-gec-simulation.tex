\Subsection{МОДЕЛИРОВАНИЕ ГЭЦ С ПРОВОДИМОСТЬЮ, ЗАВИСЯЩЕЙ ТОЛЬКО ОТ ВЫСОТЫ}

В данной части работы использовалась модель ГЭЦ, основанная на параметризации ИП, предложенной в \cite{Slyunyaev_et_al_2019} на основе идей \cite{Mareev_Volodin_2014}.
Такая параметризация использует функцию проводимости, которая кладётся зависящей лишь от высоты $\sigma(z)  = \sigma_0 \exp\qty(z/H)$ (см. ниже), и определяет вклады в ИП от каждой ячейки модельной сетки в терминах климатических параметров так, что ИП даётся выражением
\begin{equation}\label{eq:ip}
 	V = \dfrac{j_0H}{\sigma_0 S_E} \sum\limits_i \qty[\exp\qty(-\dfrac{z^1_i}{H}) - \exp\qty(-\dfrac{z^2_i}{H})] \times \dfrac{P_i S_i}{W_i} \times \theta(\varepsilon_i - \varepsilon_0),
\end{equation}
где $j_0$ --- характерная величина тока разделения зарядов в облаках с развитой электрической структурой, $\sigma_0$ и $H$ приповерхностное значение проводимости воздуха и характерный масштаб увеличения проводимости (в вычислениях полагается $H=6\,\textnormal{км}$), $S_i$ --- площадь, занимаемая $i$-ой ячейкой; $P_i$, $W_i$, $\varepsilon_i$, $z_i^1$ и $z_i^2$ --- общее количество осадков, взятой за симметричный двух часовой интервал, общее количество влаги, максимальное значение convective available potential energy (CAPE) и верхняя и нижняя границы области смешанной фазы в облаке (которые приближались высотами изотерм $0$ \textdegree C и $-38$ \textdegree C) в $i$-ом столбце соответственно, а $\varepsilon_0$ --- граничное значение CAPE, которое полагалось в расчётах $1\, \textnormal{кДж}/\textnormal{кг}$. Сумма (\ref{eq:ip}) берётся по по всем столбцам модели. Более глубокий анализ формулы (\ref{eq:ip}) производится в \cite{Ilin_et_al_2020}. Важно отметить основные идеи данной параметризации: второй множитель в формуле (\ref{eq:ip}) оценивает площадь, занимаемую облаками в каждой из ячеек модели, а третий множитель (\ref{eq:ip}) позволяет выделить столбцы с развитой конвективной активностью по критерию, основанному на значении CAPE.

Для инициализации данной модели ГЭЦ необходимо задать параметры (высоты изотерм, осадки и CAPE), которые рассчитываются при помощи климатического моделирования. Для этого была воспроизведена атмосферная динамика с помощью Weather Research and Forecasting model (WRF) за каждый третий день с 1 января 1980 года по 28 декабря 2020 года на широтно-долготной сетке 1\textdegree\texttimes1\textdegree. В качестве начальных значений для WRF использовались данные ERA5 \cite{ERA5}.

В итоге были рассчитаны значения ИП за каждый третий день в период 1980--2020. Значение величины $j_0$ в (\ref{eq:ip}) является параметром модели, оно подбиралось таким образом, чтобы среднее значение моделируемого ИП было равно $240\,\textnormal{кВ}$, что соответствует типичному значению ИП \cite{Markson_2007}.