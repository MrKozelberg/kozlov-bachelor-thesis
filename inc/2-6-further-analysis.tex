\subsection{БОЛЕЕ ДЕТАЛЬНЫЙ АНАЛИЗ И ОБЪЯСНЕНИЕ ФИЗИЧЕСКИХ МЕХАНИЗМОВ}
\label{sec:futher_analysis}

Была обнаружена связь КМД с ИП в том числе и в результатах моделирования ГЭЦ. Используя данные вкладов одиночных столбцов модели, можно проанализировать найденную связь более детально. В данном подразделе будет предпринята попытка разложить вариацию ИП на простые колебания, чтобы понять физический механизм, который стоит за наблюдаемым эффектом.

В климатологии принято вычислять ЭОФ и ГК для различных физических параметров с целью идентифицировать КМД \cite{Knutson_Weickmann_1987, Slingo_et_al_1999, Lo_Hendon_2000, Matthews_2000, Kessler_2001}. ЭОФ являются определенными собственными векторами данных, которые отражают основные пространственные паттерны, в то время как ГК являются зависящими от времени коэффициентами разложения исходных данных по базису, составленному из ЭОФ. В разделе \ref{sec:rmm} было описано, как в \cite{Wheeler_Hendon_2004} используют ЭОФ и ГК (рассчитанные для комбинированного набора данных, составленного из зональных ветров на двух высотах и OLR), чтобы ввести индекс RMM, который используется для описания КМД. В данном разделе будет применен ЭОФ-анализ к моделируемым вкладам в ИП --- это позволит свести сложную изменчивость ИП к нескольким простым осцилляциям, имеющим понятный физический смысл и легко интерпретируемым в терминах КМД.

\subsubsection{ПРЕДВАРИТЕЛЬНАЯ ОБРАБОТКА ВКЛАДОВ В ИП}
\label{sec:preliminary_processing}

Как и в \cite{Wheeler_Hendon_2004}, рассматривался приэкваториальный регион, ограниченный с севера и с юга широтами 15\textdegree~с.~ш. и 15\textdegree~ю.~ш. соответственно. Согласно результатам моделирования ГЭЦ, в среднем данный регион дает 86\% от всего ИП, что делает его ключевым в изучении связи ГЭЦ с КМД. Моделируемые вклады в ИП суммировались вдоль каждой из 360 вытянутых вдоль меридиан полосок 1\textdegree\texttimes30\textdegree\ около экватора (15\textdegree\ с. ш. -- 15\textdegree\ ю. ш. по широте и 0\textdegree~--~1\textdegree\ в. д., 1\textdegree\ в. д.~--~2\textdegree\ в. д. и так далее до 1\textdegree\ з. д. -- 0\textdegree\ по долготе), в результате чего получался один набор данных длиной 360 для каждого моделируемого дня. Такое суммирование упрощает сложный процесс эволюции вкладов в ИП в течение цикла КМД путем исключения из рассмотрения меридиональной структуры, которая показана на рис. \ref{fig:map_of_contributions}. Так как первостепенным в КМД является процесс переноса с запада на восток конвективной структуры, то в первом приближении можно рассматривать лишь долготную структуру вкладов для изучения связей ГЭЦ с КМД.

КМД всегда происходит одновременно с прочей конвективной изменчивостью, поэтому важно некоторым образом выявить во вкладах в ИП и удалить большую часть изменчивости, не связанной с КМД. Похожий подход применялся и в \cite{Wheeler_Hendon_2004} до расчета ЭОФ.

Прежде всего удалялась связь ГЭЦ с ЭНЮК \cite{Harrison_et_al_2011, Slyunyaev_et_al_2021c}. Для этих целей вычислялся коэффициент линейной регрессии между средними за день значениями вкладов в ИП от приэкваториальных вытянутых вдоль меридиан полос 1\textdegree\texttimes30\textdegree\ на разной долготе и температурой поверхности океана (ТПО), усреднённой по региону Niño 3.4 (5\textdegree\ с. ш. -- 5\textdegree\ ю. ш. и 120\textdegree\ в. д. -- 170\textdegree\ в. д.); температура в данном регионе хорошо характеризует ЭНЮК. Из каждой полосы 1\textdegree\texttimes30\textdegree\ вычиталась найденная линейная связь с ТПО региона Niño 3.4, при этом сохранялось долговременное среднее значение (точнее, из вкладов в ИП вычитались значения, предсказанные линейной аппроксимацией на основе ТПО региона Niño 3.4 в соответствующие дни, при этом постоянный член такой аппроксимации регулировался постоянный таким образом, чтобы среднее значение вклада данной полосы 1\textdegree\texttimes30\textdegree, вычисленное за 41 год моделирования, не изменилось).

После этого из вкладов удалялась сезонная вариация \cite{Adlerman_Williams_1996}. Для этого было применено дискретное преобразование Фурье к временному ряду вкладов каждой из полос 1\textdegree\texttimes30\textdegree, значения Фурье-спектра, отвечающие первым четырем гармоникам сезонного цикла (то есть те, которые соответствовали периоду $T=365.25\, \textnormal{дней}$, $T/2$, $T/3$ и $T/4$), клались равными нулю, затем производилось обратное дискретное преобразование Фурье. Стоит отметить, что данную  операцию следует проводить после удаления связи вкладов с ЭНЮК, ведь ТПО региона Niño 3.4 имеет свои собственные сезонные гармоники, поэтому вычитание из вкладов величин, пропорциональных ТПО региона Niño 3.4, приводит к усилению сезонных гармоник вкладов в ИП. Кроме того, следует отметить, что вычитание сезонных гармоник не изменяет среднее значение, так как долговременное среднее значение каждой гармоники равно нулю.

Рис. \ref{fig:eq_var}{a} показывает, как вариация ИП по фазам КМД изменяется после перехода ко вкладам приэкваториального региона (15\textdegree\ с. ш.~--~15\textdegree\ ю. ш.) и удаления изменчивости, не связанной с КМД, согласно алгоритму, описанному выше. Чтобы сделать сравнение более наглядным, были введены аномалии с помощью вычитания средних значений за долгий период времени из каждого ряда данных (240 кВ для ИП и около 207 кВ для вклада приэкваториального региона). Видно, что две вариации близки друг к другу, что подтверждает тот факт, что ключевым регионом при изучении связи ГЭЦ с КМД является приэкваториальный.

\begin{figure}[htbp]
    \centering
    \includegraphics[width=\textwidth]{figures/equatorial_variations.pdf}
    \caption{(a): Аномалии ИП и вклада в ИП приэкваториального региона (15\textdegree\ с. ш.~--~15\textdegree~ю.~ш.) в различные фазы КМД. Аномалии вычислялись как отклонения от долговременных средних значений. Из вклада приэкваториального региона была убрана изменчивость, имеющая отношение к ЭНЮК и сезонному циклу, как описано в разделе \ref{sec:preliminary_processing}. (b): Аномалии вклада приэкваториального региона в ИП (то же самое, что и на рисунке (a)) и та часть аномалии вклада приэкваториального региона в ИП, которая соответствует разложению вектора вкладов экваториальных вытянутых вдоль меридианов полос 1\textdegree\texttimes30\textdegree\ лишь по первым трём ЭОФ.}
    \label{fig:eq_var}
\end{figure}

\subsubsection{ВЫЧИСЛЕНИЕ ЭОФ И ГК ДЛЯ ВКЛАДОВ В ИП}

После вычитания связей вкладов с ЭНЮК и сезонным циклом получается временной ряд 360-мерных векторов (один вектор для каждого моделируемого дня), содержащих вклады в ИП различных вытянутых вдоль меридианов полос 1\textdegree\texttimes30\textdegree, отвечающих различным долготам. Такой вектор для дня $d$ будет обозначаться через
\begin{equation}
    \vb{V}(d) = \qty(V_1(d),\, \ldots,\, V_{360}(d)).
\end{equation}
Чтобы ЭОФ и ГК были корректно вычислены, требуется перейти к вектору, каждая компонента которого имеет среднее значение, равное нулю. Таким вектором будет 
\begin{equation}
    \vb{U}(d) = \qty(U_1(d),\, \ldots,\, U_{360}(d)),
\end{equation}
где $U_j(d) = V_i(d) - \langle V_i(d) \rangle$ (угловые скобки обозначают долгосрочное усреднение по $d$).

Ниже будет описан процесс вычисления ЭОФ \cite[Гл. 6]{Zhang_Moore_2015} и приведена его практическая реализация применительно к $\vb{U}(d)$. Можно понимать компоненты данного вектора $U_1(d),\, \ldots,\, U_{360}(d)$ как координаты 360-мерного вектора в стандартном базисе в пространстве $\mathbb{R}^{360}$ $\vb{e}^{(1)},\, \ldots,\, \vb{e}^{(360)}$. Главная идея ЭОФ-анализа заключается в нахождении другого ортонормированного базиса $\vb{f}^{(1)},\, \ldots,\, \vb{f}^{(360)}$ такого, что его первые компоненты (то есть  $\vb{f}^{(1)},\, \vb{f}^{(2)}$ и несколько следующих) в некотором смысле описывали бы наибольшую часть изменчивости $\vb{U}(d)$. Нахождение такого базиса позволит устроить разложение
\begin{equation}
    \label{eq:decomp}
    \vb{U}(d) = \sum\limits_{j=1}^{360} U_j(d) \vb{e}^{(j)} = \sum\limits_{j=1}^{360} C_j(d) \vb{f}^{(j)}.
\end{equation}
Элементы нового базиса $\vb{f}^{(1)},\, \ldots,\, \vb{f}^{(360)}$ называются ЭОФ (эмпирическими ортогональными функциями), а временные коэффициенты разложения по нему $C_1(d),\, \ldots,\, C_{360}(d)$ называются ГК (главными компонентами). Стоит заметить, что $\vb{f}^{(j)}$ является 360-мерным вектором и в стандартном базисе $\vb{e}^{(1)},\, \ldots,\, \vb{e}^{(360)}$ записывается в виде
\begin{equation}
    \vb{f}^{(j)} = \qty({f}_1^{(j)},\, \ldots,\, {f}_{360}^{(j)}),
\end{equation}
то есть $\vb{f}^{(j)}$ можно понимать как функцию долготы (долготе отвечает нижний индекс).

Ниже будет дано строгое математическое определение ЭОФ и ГК, для чего будет использоваться скалярное произведение в пространстве $\mathbb{R}^{360}$, определенное следующим образом:
\begin{equation}
    \qty(\vb{a},\, \vb{b}) = \sum\limits_{j=1}^{360} a_j b_j,\; \text{где}\; \vb{a} = \sum\limits_{j=1}^{360} a_j \vb{e}^{(j)},\; \vb{b} = \sum\limits_{j=1}^{360} b_j \vb{e}^{(j)}.
\end{equation}
Под нормой будет пониматься 
\begin{equation}
    \norm{\vb{a}} = \sqrt{\sum\limits_{j=1}^{360} a_j},\; \text{где}\; \vb{a} = \sum\limits_{j=1}^{360} a_j \vb{e}^{(j)}.
\end{equation}
То есть будут использоваться стандартное скалярное произведение в пространстве $\mathbb{R}^{360}$ и стандартная евклидова норма. Первая ЭОФ $\vb{f}^{(1)}$ определяется как такой единичный вектор ($\norm{\vb{f}^{(1)}}=1$), который минимизирует величину
\begin{equation}
    \epsilon_1 = \expval{\norm{\vb{U}(d) - (\vb{U}(d),\, \vb{f}^{(1)})\vb{f}^{(1)}}^2}.
\end{equation}
Похожим образом определяется вторая ЭОФ $\vb{f}^{(2)}$. Согласно определению, $\vb{f}^{(2)}$ есть такой единичный вектор ($\norm{\vb{f}^{(2)}}=1$), ортогональный $\vb{f}^{(1)}$ ($\qty(\vb{f}^{(1)},\, \vb{f}^{(2)})=0$), который минимизирует величину
\begin{equation}
    \epsilon_2 = \expval{\norm{\vb{U}(d) - (\vb{U}(d),\, \vb{f}^{(1)})\vb{f}^{(1)} - (\vb{U}(d),\, \vb{f}^{(2)})\vb{f}^{(2)}}^2}.
\end{equation}
Следующие ЭОФ определяются аналогичным образом: $j$-ая ЭОФ $\vb{f}^{(j)}$ есть такой единичный вектор ($\norm{\vb{f}^{(j)}}=1$), ортогональный $\vb{f}^{(1)},\, \ldots,\, \vb{f}^{(j-1)}$, который минимизирует величину
\begin{equation}
    \epsilon_j = \expval{\norm{\vb{U}(d) - \sum\limits_{k=1}^{j} \qty(\vb{U}(d),\, \vb{f}^{(k)})\vb{f}^{(k)}}^2}.
\end{equation}
ГК определяются как координаты $\vb{U}(d)$ в новом базисе: $j$-ая ГК $C_j(d)$ задается формулой
\begin{equation}
    C_j(d) = (\vb{U}(d),\, \vb{f}^{(j)}).
    \label{eq:pc}
\end{equation}

Можно придать ЭОФ более наглядный смысл. Пользуясь тем, что $\expval{U_j(d)}=0$ для всех $j$, не трудно показать, что
\begin{equation}
    \epsilon_1 = \expval{\norm{\vb{U}(d)}^2} - \expval{\qty(\vb{U}(d),\, \vb{f}^{(1)})^2} = \sum\limits_{j=1}^{360} \mathrm{Var}\, U_j(d) - \mathrm{Var}\, C_1(d),
\end{equation}
где через $\mathrm{Var}$ обозначена дисперсия. Аналогично можно получить
\begin{equation}
    \epsilon_j = \expval{\norm{\vb{U}(d)}^2} - \sum\limits_{k=1}^{j} \expval{\qty(\vb{U}(d),\, \vb{f}^{(k)})^2} = \sum\limits_{i=1}^{360} \mathrm{Var}\, U_i(d) - \sum\limits_{k=1}^{j}\mathrm{Var}\, C_j(d).
\end{equation}
Отсюда видно, что первая ЭОФ $\vb{f}^{(1)}$ максимизирует $\mathrm{Var}\, C_1(d)$, вторая ЭОФ $\vb{f}^{(2)}$ выбирается в оставшихся измерениях таким образом, чтобы максимизировать $\mathrm{Var}\, C_2(d)$, и так далее.

Численное нахождение ЭОФ основано на поиске собственных векторов и собственных значений ковариационной матрицы $\Sigma$, элементы которой определяются как
\begin{equation}
    \Sigma_{ij} = \mathrm{Cov}\, \qty(U_i(d), U_j(d)).
\end{equation}
Можно показать, что ковариационная матрица является симметричной и положительно определенной \cite[Утв. 6.1]{Zhang_Moore_2015}; отсюда следует, что собственные значения такой матрицы будут действительными положительными числами. Нетрудно показать, что первая ЭОФ является собственным вектором матрицы $\Sigma$, отвечающим наибольшему собственному значению, вторая ЭОФ является собственным вектором матрицы $\Sigma$, отвечающим второму по величине собственному значению, и так далее \cite[Tеор. 6.1 и 6.3]{Zhang_Moore_2015}. После нахождения ЭОФ по формуле \eqref{eq:pc} находят ГК.

Принято вводить такую величину, как объясняемая дисперсия. Она определяется для каждой из ЭОФ и в некотором смысле показывает, какую часть дисперсии объясняет данная ЭОФ. Если говорить строго, то объясняемая дисперсия $j$-ой ЭОФ определяется как $\mathrm{Var}\, C_j(d)$. Можно доказать, что объясняемая дисперсия $j$-ой ЭОФ равна $j$-ому собственному значению ковариационной матрицы $\Sigma$ \cite[Утв. 6.2]{Zhang_Moore_2015}.

Таким образом, переход к базису, образованному из ЭОФ, позволяет устроить разложение \eqref{eq:decomp}, то есть представить вектор $\vb{U}(d)$ в виде суммы взаимно ортогональных компонент $C_1(d)\vb{f}^{(1)},\, \ldots,\, C_{360}(d) \vb{f}^{(360)}$, первые несколько из которых отвечают за наибольшую часть дисперсии данных.

Основываясь на \eqref{eq:decomp}, можно записать аномалию суммарного вклада в ИП от приэкваториального региона (без изменчивости, связанной с ЭНЮК и сезонным циклом) за день $d$ в виде
\begin{equation}
\label{eq:u_eq}
    U_{\mathrm{eq}} (d) = \sum\limits_{j=1}^{360} U_j(d) = \sum\limits_{j,i=1}^{360} C_i(d) f_j^{(i)}.
\end{equation}
Если же вместо вектора $\vb{U}(d)$ перейти к рассмотрению его главной части, которая описывается первыми несколькими ЭОФ, то есть к вектору
\begin{equation}
\label{eq:u_tilde}
    \Tilde{\vb{U}}(d) = C_1(d) \vb{f}^{(1)} + C_2(d) \vb{f}^{(2)} + C_3(d) \vb{f}^{(3)},
\end{equation}
где учтены лишь первые три ЭОФ, то соответствующая такому вектору аномалия вклада будет даваться выражением
\begin{equation}
\label{eq:u_tilde_eq}
    \Tilde{U}_{\mathrm{eq}}(d) =\sum\limits_{j=1}^{360} \qty{C_1(d) {f}_j^{(1)} + C_2(d) {f}_j^{(2)} + C_3(d) {f}_j^{(3)}} = \sum\limits_{j=1}^{360} \sum\limits_{i=1}^{3} C_i(d) f_j^{(i)}.
\end{equation}
На рис. \ref{fig:eq_var}{b} проводится сравнение средних значений $U_{\mathrm{eq}} (d)$ и $\Tilde{U}_{\mathrm{eq}}(d)$ в различные фазы КМД. Видно, что для воспроизведения синусоидальной вариации, которая присутствует во всей сумме, хватает учета лишь первых трех ЭОФ. Это позволяет рассматривать \eqref{eq:u_tilde} и \eqref{eq:u_tilde_eq} вместо \eqref{eq:decomp} и \eqref{eq:u_eq} с целью исследования физического механизма, обеспечивающего наблюдаемую вариацию ИП по фазам КМД.

Чтобы исследовать более детально вариацию $\Tilde{U}_{\mathrm{eq}}(d)$ на масштабе КМД, представленную на рис. \ref{fig:eq_var}{b}, следует разложить такую вариацию на отдельные части, отвечающие различным ЭОФ. Рис. \ref{fig:eofs_and_pcs}{a} показывает каждое из трех слагаемых \eqref{eq:u_tilde}, усредненное по различным фазам КМД. Вклад в ИП, отвечающий ЭОФ1, равен
\begin{equation}
\label{eq:u1}
    U_\mathrm{eq}^{(1)} (d) = C_1(d) \sum\limits_{j=1}^{360} {f}_j^{(1)}
\end{equation}
и имеет на масштабе КМД синусоидальную вариацию с максимумом в четвертой фазе и минимумом в восьмой фазе. Вклады в ИП, отвечающие ЭОФ2 и ЭОФ3, равны
\begin{equation}
\label{eq:u23}
    \begin{split}
        U_\mathrm{eq}^{(2)} (d) = C_2(d) \sum\limits_{j=1}^{360} {f}_j^{(2)},\\
        U_\mathrm{eq}^{(3)} (d) = C_3(d) \sum\limits_{j=1}^{360} {f}_j^{(3)}
    \end{split}
\end{equation}
и имеют близкие синусоидальные вариации на масштабе КМД с максимумом во второй фазе и минимумом в шестой.

\begin{figure}[htbp]
    \centering
    \includegraphics[width=\textwidth]{figures/eofs_and_pcs.pdf}
    \caption{(a): Изменчивость вклада в ИП экваториального региона (15\textdegree\ с. ш.~--~15\textdegree~ю.~ш.) в различные фазы КМД, отвечающая каждой из первых трех ЭОФ в отдельности. (b): Пространственная структура первых трех ЭОФ. (c): ГК, отвечающие первым трем ЭОФ, усредненные за различные фазы КМД. (d)--(f): То же, что и (a)--(c) для повернутых ЭОФ. Числа в легендах на рисунках (b) и (e) обозначают величину объясняемой дисперсии данной ЭОФ.}
    \label{fig:eofs_and_pcs}
\end{figure}

Рис. \ref{fig:eofs_and_pcs}{b} демонстрирует долготные профили первых трех ЭОФ (ЭОФ1 $\vb{f}^{(1)}$, ЭОФ2 $\vb{f}^{(2)}$ и ЭОФ3 $\vb{f}^{(3)}$). ЭОФ1 описывает значительную аномалию во вкладах в ИП, расположенную между долготами 80\textdegree\ в. д. и 180\textdegree\ и имеющую максимум на 150\textdegree\ в. д.; ЭОФ2 и ЭОФ3 больше всего отличны от нуля между 50\textdegree\ в. д. и 170\textdegree\ з. д., они обе имеют сильный минимум между 160\textdegree\ в. д. и 170\textdegree\ в. д., кроме того, около 90\textdegree\ з. д. они достигают экстремумов разного знака.

Для большей наглядности на рис. \ref{fig:eofs_and_pcs}{c} приведены усредненные по различным фазам КМД ГК, отвечающие первым трем ЭОФ (то есть ГК1 $C_1(d)$, ГК2 $C_2(d)$ и ГК3 $C_3(d)$). Важно понимать, что усредненные ГК пропорциональны усредненным аномалиям вкладов в ИП, отвечающим различным ЭОФ (см. рис. \ref{fig:eofs_and_pcs}{a}); это прямо следует из выражений \eqref{eq:u1} и \eqref{eq:u23}, а так же того факта, что ЭОФ лишь описывают пространственную структуру вкладов и не меняются со временем.

\subsubsection{ПОВОРОТ БАЗИСА ЭОФ}
\label{sec:rot_eof}

Из рис. \ref{fig:eofs_and_pcs}{c} видно, что ГК2 и ГК3 изменяются на масштабе КМД в противофазе. В то же время соответствующие им ЭОФ, ЭОФ2 и ЭОФ3, как функции долготы крайне схожи на востоке около 150\textdegree\ в. д., но сильно расходятся около 90\textdegree\ в. д. (см. рис. \ref{fig:eofs_and_pcs}{b}). Это наводит на мысль о возможности введения более простых пространственных паттернов; вместо ЭОФ2 $\vb{f}^{(2)}$ и ЭОФ3 $\vb{f}^{(3)}$ можно перейти к
\begin{equation}
    \vb{f}^{(2')} = \dfrac{\vb{f}^{(2)} - \vb{f}^{(3)}}{\sqrt{2}},\; \vb{f}^{(3')} = \dfrac{\vb{f}^{(2)} + \vb{f}^{(3)}}{\sqrt{2}},
\end{equation}
которые будут обозначаться через ЭОФ2$^\prime$ и ЭОФ3$^\prime$ соответственно. Аналогично, вместо ГК2 $C_2(d)$ и ГК3 $C_3(d)$ следует перейти к 
\begin{equation}
    C_{2'}(d) = \dfrac{C_2(d) - C_3(d)}{\sqrt{2}} ,\;  C_{3'}(d)  = \dfrac{C_2(d) + C_3(d)}{\sqrt{2}},
\end{equation}
которые будут обозначаться через ГК2$^\prime$ и ГК3$^\prime$. Такая техника называется поворотом ЭОФ \cite[Гл. 6]{Zhang_Moore_2015}; грубо говоря, данная техника позволяет подогнать автоматически выбранный базис к решаемой проблеме (стоит заметить, что оригинальные ЭОФ были рассчитаны без каких-либо предположений о КМД, его временных и пространственных масштабов).

В повернутом базисе ЭОФ выражения \eqref{eq:u_tilde} и \eqref{eq:u_tilde_eq} примут вид
\begin{equation}
    \Tilde{\vb{U}}(d) = C_1(d) \vb{f}^{(1)} + C_{2'}(d) \vb{f}^{(2')} + C_{3'}(d) \vb{f}^{(3')},
\end{equation}
\begin{equation}
\label{eq:u_tilde_eq_new}
    \Tilde{U}_{\mathrm{eq}}(d) =\sum\limits_{j=1}^{360} \qty{C_1(d) {f}_j^{(1)} + C_{2'}(d) {f}_j^{(2')} + C_{3'}(d) {f}_j^{(3')}},
\end{equation}
выражение \eqref{eq:u1} останется прежним, вместо выражения \eqref{eq:u23} следует использовать
\begin{equation}
    \begin{split}
        U_\mathrm{eq}^{(2')} (d) = C_{2'}(d) \sum\limits_{j=1}^{360} {f}_j^{(2')},\\
        U_\mathrm{eq}^{(3')} (d) = C_{3'}(d) \sum\limits_{j=1}^{360} {f}_j^{(3')}.
    \end{split}
\end{equation}

Рис. \ref{fig:eofs_and_pcs}{e} и рис. \ref{fig:eofs_and_pcs}{f} являются аналогами рис. \ref{fig:eofs_and_pcs}{b} и \ref{fig:eofs_and_pcs}{c} для повернутых ЭОФ (ЭОФ1 $\vb{f}^{(1)}$, ЭОФ2$^\prime$ $\vb{f}^{(2')}$ и ЭОФ3$^\prime$ $\vb{f}^{(3')}$) и новых ГК (ГК1 $C_1(d)$, ГК2$^\prime$ $C_{2'}(d)$ и ГК3$^\prime$ $C_{3'}(d)$). Из сравнения рис. \ref{fig:eofs_and_pcs}{e} и \ref{fig:eofs_and_pcs}{b} видно, что пространственная структура новых ЭОФ стала действительно проще по сравнению со структурой оригинальных ЭОФ. ЭОФ2$^\prime$ описывает в значительной степени аномалию вкладов в ИП, расположенных между 50\textdegree\ в. д. и 120\textdegree\ в. д. с одним крупным максимумом в районе 90\textdegree\ в. д., в то время как ЭОФ3$^\prime$ описывает более сложную структуру вкладов с главным экстремумом в районе 170\textdegree\ в. д. и несколькими локальными экстремумами. Если посмотреть на средние значения ГК в течение фаз КМД (см. рис. \ref{fig:eofs_and_pcs}{f}), то видно, что в среднем значения ГК3$^\prime$ меньше по абсолютной величине, чем значения ГК2$^\prime$ и ГК1. Поэтому в дальнейшем можно удержать лишь ЭОФ1 и ЭОФ2$^\prime$, пренебрегая ЭОФ3$^\prime$.

Далее следует рассмотреть аномалии вклада экваториального региона в ИП. На рис.~\ref{fig:eofs_and_pcs}{d} показаны три слагаемых разложения \eqref{eq:u_tilde_eq_new}, усредненных по различным фазам КМД (аналогично рис. \ref{fig:eofs_and_pcs}{a}, где были изображены слагаемые разложения \eqref{eq:u_tilde_eq}). Легко видеть, что слагаемое $U_\mathrm{eq}^{(3')} (d)$, относящееся к ЭОФ3$^\prime$, пренебрежимо мало, что не удивительно, ведь не только ГК3$^\prime$ мала по сравнению с двумя прочими ГК, но и сумма компонент ЭОФ3$^\prime$ $\sum_{j=1}^{360} {f}_j^{(3')}$ мала (см. кривую ЭОФ3$^\prime$ на рис.~\ref{fig:eofs_and_pcs}{e}). Остальные слагаемые $U_\mathrm{eq}^{(1)} (d)$ и $U_\mathrm{eq}^{(2')} (d)$, отвечающие ЭОФ1 и ЭОФ2$^\prime$ соответственно, имеют синусоидальные вариации по фазам КМД с близкими амплитудами. Одна вариация обгоняет другую на 1 четверть периода (2 фазы КМД). Вариация аномалии ИП, показанная на рис. \ref{fig:eq_var}, может быть во многом приближена суммой двух таких базовых колебаний.

\subsubsection{ДОЛГОТНАЯ СТРУКТУРА БАЗОВЫХ КОЛЕБАНИЙ}

Поворот базиса, составленного из ЭОФ, позволил свести наблюдаемую вариацию ИП по фазам КМД (см. рис. \ref{fig:variations}{a}) к суперпозиции двух базовых колебаний, задаваемых усредненными по фазам КМД величинами $U_\mathrm{eq}^{(1)} (d)$ и $U_\mathrm{eq}^{(2')} (d)$ (см. рис. \ref{fig:eofs_and_pcs}{d}). Если рассуждать в терминах пространственных паттернов, то величины $U_\mathrm{eq}^{(1)} (d)$ и $U_\mathrm{eq}^{(2')} (d)$ определяются как суммы компонент векторов
\begin{equation}
    \vb{U}^{(1)}(d) = C_1(d) \vb{f}^{(1)}
    \label{eq:u_eof1}
\end{equation}
и
\begin{equation}
    \vb{U}^{(2')}(d) = C_{2'}(d) \vb{f}^{(2')}
    \label{eq:u_eof2'}
\end{equation}
вдоль долготных индексов. В данном разделе будет рассмотрена долготная структура данных векторов в различные фазы КМД (то есть суммирование по долготным индексам проводиться не будет).

Левый столбец рис. \ref{fig:longitudinal_structure} показывает усредненную долготную структуру аномалий экваториальных вкладов в ИП (то есть $\vb{U}(d)$ из выражения \eqref{eq:decomp}) для различных фаз КМД. Это буквально одномерная версия рис. \ref
{fig:map_of_contributions}: аномалии вкладов клеток 1\textdegree\texttimes1\textdegree\ в ИП, показанные на рис. \ref{fig:map_of_contributions}, составляют аномалии вкладов от вытянутых вдоль меридианов полос 1\textdegree\texttimes30\textdegree, показанные в левом столбце рис. \ref{fig:longitudinal_structure}; именно поэтому красные и синие области на рис. \ref{fig:map_of_contributions}, обозначающие положительные и отрицательные аномалии вкладов соответственно, в основном совпадают с интервалами положительных и отрицательных аномалий в левом столбце рис. \ref{fig:longitudinal_structure}.

\begin{figure}[htbp]
    \centering
    \includegraphics[width=\textwidth]{figures/longitudinal_structure.pdf}
    \caption{Левый столбец: Аномалии вкладов в ИП от вытянутых вдоль меридианов полос 1\textdegree\texttimes30\textdegree\ (15\textdegree\ с. ш.~--~15\textdegree\ ю. ш.) при различных долготах в течение каждой из восьми фаз КМД. Аномалии вычисляются по отношению к долгосрочным средним значениям. Изменчивость, связанная с ЭНЮК и сезонным циклом, удалена из данных (см. раздел \ref{sec:preliminary_processing}) перед построением графиков. Средний столбец: То же самое за тем исключением, что была оставлена лишь изменчивость, относящаяся к ЭОФ1 и ЭОФ2$^\prime$ (см. раздел \ref{sec:rot_eof}). Правый столбец: Изменчивость, отвечающая ЭОФ1 и ЭОФ2$^\prime$, показана раздельно.}
    \label{fig:longitudinal_structure}
\end{figure}

Средний столбец рис. \ref{fig:longitudinal_structure} показывает, как левый столбец преобразуется, если вместо разложения по всем ЭОФ, которое может быть записано в следующем виде:
\begin{equation}
    \vb{U}(d) = C_1(d) \vb{f}^{(1)} + C_{2'}(d) \vb{f}^{(2')} + C_{3'}(d) \vb{f}^{(3')} + \sum\limits_{j=4}^{360} C_j(d) \vb{f}^{j},
\end{equation}
удержать лишь первые два слагаемых. Сравнивая средний и левый столбцы, можно заключить, что учет лишь слагаемых, отвечающих ЭОФ1 и ЭОФ2$^\prime$, сохраняет главные паттерны в динамике вкладов в ИП на масштабах КМД.

Правый столбце рис. \ref{fig:longitudinal_structure} разлагает то, что было показано в среднем столбце, на две компоненты, показывая раздельно усредненные аномалии вкладов, отвечающие ЭОФ1 и ЭОФ2$^\prime$. Можно видеть, что сложная динамика вкладов, наблюдаемая в среднем столбце, оказывается суперпозицией двух простых колебаний, долготная структура которых постоянная и задается ЭОФ1 и ЭОФ2$^\prime$ (см. рис. \ref{fig:eofs_and_pcs}{e}). Амплитуда двух выделенных в правом столбце рис. \ref{fig:longitudinal_structure} колебаний имеет синусоидальную вариацию по фазам КМД и определяется ГК1 и ГК2$^\prime$ (см. рис. \ref{fig:eofs_and_pcs}{f}).

Таким образом, компонента \eqref{eq:u_eof1}, отвечающая ЭОФ1, в основном локализована между 80\textdegree\ в. д. и 180\textdegree\ с максимумом на долготе 150\textdegree\ в. д.; такая структура в среднем периодичным образом меняется на масштабе КМД, достигая наибольшего значения амплитуды в четвертой фазе и минимального значения амплитуды в восьмой фазе. Компонента \eqref{eq:u_eof2'}, отвечающая ЭОФ2$^\prime$, сконцентрирована между 50\textdegree\ в. д. и 120\textdegree\ в. д. и имеет основной максимум на долготе около 90\textdegree в. д.; в среднем такая структура имеет максимум амплитуды во второй фазе КМД и минимум в шестой фазе.

\subsubsection{СВЯЗЬ МЕЖДУ ИП И КМД}

Подводя итог вышеописанному, важно отметить, что удалось выделить два базовых колебания вкладов в ИП на масштабе КМД (см. правый столбец рис. \ref{fig:longitudinal_structure}), одно из которых происходит над Морским континентом (80\textdegree\ в. д. -- 180\textdegree), а второе --- над Индийским океаном (50\textdegree\ в. д. -- 120\textdegree\ в. д.). Второе колебание опережает первое на четверть периода, то есть на 2 фазы КМД; при суммировании вкладов, относящихся к различным долготам, эти два колебания достигают одинаковой амплитуды (см. рис. \ref{fig:eofs_and_pcs}{d}) и вме{}сте дают похожую на синус вариацию с максимумом в третьей фазе и минимумом в седьмой. Это объясняет большую часть изменчивости ИП на масштабе КМД (см. рис. \ref{fig:eq_var}) и тем самым редуцирует задачу объяснения наблюдаемой вариации ИП по фазам КМД (см. рис. \ref{fig:variations}{a}) к пониманию физической природы вышеупомянутой пары базовых колебаний.

Следует напомнить, что двумерный индекс RMM, который используется для численного описания КМД и на основе которого выделяются восемь фаз КМД, был введен в работе \cite{Wheeler_Hendon_2004} на основе ЭОФ-анализа, примененного для комбинированного поля данных, составленного из данных по зональным ветрам (которые являются характеристикой циркуляции) и OLR (который служит для описания конвекции; см. раздел \ref{sec:rmm}). Если посмотреть на выделенные в работе \cite{Wheeler_Hendon_2004} ЭОФ (см. рис. \ref{fig:wh04_fig1}), то нетрудно увидеть, что выделенная в \cite{Wheeler_Hendon_2004} ЭОФ1 отвечает отрицательной аномалии OLR в интервале долгот 70\textdegree\ в. д. -- 180\textdegree\ с пологим минимумом между 120\textdegree\ в. д. и 140\textdegree\ в. д. (см. непрерывную кривую на левой части рис. \ref{fig:wh04_fig1}), в то время как ЭОФ2 в данной работе описывает положительную аномалию OLR, простирающуюся по долготе между 40\textdegree\ в. д. и 110\textdegree\ в. д. с максимумом около 80\textdegree\ в. д. (см. непрерывную кривую на правой части рис. \ref{fig:wh04_fig1}). Учитывая тот факт, что моделируемые вклады в ИП, рассчитываемые согласно \eqref{eq:ip}, тоже отражают географическое распределение глубокой конвекции, то естественно искать соответствие между выделенными в настоящей работе ЭОФ на основе вкладов в ИП и теми частями ЭОФ, выделенными в \cite{Wheeler_Hendon_2004}, которые относятся к OLR. Сравнивая два набора ЭОФ, можно прийти к выводу, что выделенные в настоящей работе ЭОФ1 и ЭОФ2$^\prime$ описывают примерно те же паттерны конвекции, что и ЭОФ, выделенные в \cite{Wheeler_Hendon_2004}. Если говорить точнее, то ЭОФ1 и ЭОФ2 для OLR приближенно совпадают с ЭОФ1 и с взятой с обратным знаком ЭОФ2$^\prime$ для вкладов в ИП (при проведении данной аналогии было учтено, что положительная аномалия OLR в экваториальном регионе соответствует отрицательной аномалии вкладов). 

Первые две ГК, выделенные в \cite{Wheeler_Hendon_2004}, являются RMM1 и RMM2. Проводя аналогию, начатую выше, и на ГК, стоит ожидать, что RMM1 и RMM2 соответствуют ГК1 $C_1(d)$ и взятой с обратным знаком ГК2$^\prime$ $-C_{2'}(d)$. В соответствии с определением фазы КМД (см. рис. \ref{fig:variations}{b}), RMM1 содержит близкую к синусоидальной вариацию по фазам КМД с максимумом между четвертой и пятой фазами, а RMM2 тоже имеет близкую к синусоидальной вариацию по фазам с максимумом между шестой и седьмой фазами. В то же время, $C_1(d)$ и $-C_{2'}(d)$ достигают максимумов в четвертой и шестой фазах соответственно. Это означает, что на масштабе КМД выделенные в настоящей работе ГК1 и ГК2$^\prime$ действительно описывают примерно те же колебания конвекции, что и RMM1 и RMM2, правда с небольшим фазовым сдвигом, который может быть объяснен различием в типе данных, на основе которых выделялись рассматриваемые ЭОФ.

Таким образом, видно, что два базовых колеблющихся паттерна вкладов в ИП примерно совпадают на масштабе КМД с двумя основными колеблющимися паттернами OLR, которые были выделены в \cite{Wheeler_Hendon_2004}, чтобы определить индекс RMM. Если говорить более общо, то следует отметить, что выделенные в настоящей работе ЭОФ для вкладов в ИП оказываются близки к характерным паттернам OLR, выделяемым в исследованиях КМД. Например, первые три ЭОФ для OLR, которые выделяются в \cite[рис. 2]{Kessler_2001}, оказываются похожи на ЭОФ1, ЭОФ2 и ЭОФ3, которые были выделенные в данной работе сначала (до поворота базиса ЭОФ; см. рис. \ref{fig:eofs_and_pcs}{b}). Кроме того, двумерные ЭОФ для OLR, выделяемые в \cite{Lo_Hendon_2000, Matthews_2000}, близки к тем паттернам, что и исходные ЭОФ1 и ЭОФ2.

Можно заключить, что два базовых колебания в аномалиях вкладов в ИП, которые обеспечивают большую часть изменчивости исходного ИП на масштабах КМД, примерно соответствуют тем колеблющимся паттернам конвекции, которые являются типичными для КМД, в частности тем, которые используются при расчете индекса RMM. Если брать во внимание тот факт, что компоненты индекса RMM имеют синусоидальную вариацию с изменением КМД, то не кажется таким удивительным, что и ИП имеет синусоидальную вариацию относительно фазы. В некотором смысле вклады в ИП являются мерой глубокой конвекции, поэтому типичный для КМД перенос конвективной структуры с запада на восток находит свое отражение как в ИП, так и в ГЭЦ в целом. 

Выше было показано, что вклады в ИП являются хорошими показателями глубокой конвекции (об этом говорит и само их определение, и тот факт, что они содержат в себе динамику, характерную для КМД), что позволяет чисто формально ввести новый индекс КМД на основе вкладов в ИП. По сути такой индекс уже был рассчитан выше --- ГК1 и ГК2$^\prime$ могут служить неплохой заменой RMM1 и отрицательной RMM2.