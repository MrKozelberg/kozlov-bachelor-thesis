\Anonsection{ВВЕДЕНИЕ}

В земной атмосфере протекают процессы, формирующие климат Земли, что делает изучение атмосферы критически важным для человека. Атмосферное электричество относится к числу наиболее актуальных направлений в науке, изучающей физику атмосферы Земли. Главной задачей атмосферного электричества является разработка не противоречащей эксперименту модели распределения крупномасштабного электрического поля в атмосфере планеты. Возможно, решение такой задачи позволит создать более точные климатические модели, учитывающие взаимодействие атмосферной динамики с электрическим окружением Земли.

Ключевым понятием атмосферного электричества является глобальная электрическая цепь (ГЭЦ) \cite{Williams_Mareev_2014}. ГЭЦ представляет собой распределённый токовый контур, образованный слоем воздуха между землёй и ионосферой. Выделяют два типа ГЭЦ: переменного тока и постоянного. В ГЭЦ первого типа источниками выступают молниевые разряды облако-земля, в ГЭЦ постоянного тока источниками являются токи разделения зарядов в облаках с развитой электрической структурой \cite{Williams_Mareev_2014}. Всюду ниже будет рассматриваться ГЭЦ постоянного тока. Интенсивность ГЭЦ характеризуется ионосферным потенциалом (ИП), который определяется как разность потенциалов на ионосфере и на земле. Замечательной особенностью ИП является то, что он в первом приближении не зависит от географического места измерения. Однако, ИП в последние десятилетия не измеряется (из-за дороговизны таких измерений) и служит скорее для теоретического моделирования ГЭЦ. Экспериментально измеряется градиент потенциала (ГП) электрического поля Земли, который в дни хорошей погоды пропорционален ИП. ГП в отличие от ИП подвержен множеству локальных эффектов, модулирующих ГП и осложняющих интерпретацию результатов измерений.

ГЭЦ объединяет в себе области плохой погоды, где в среднем электрические токи поднимаются вверх от поверхности земли к ионосфере, и области хорошей погоды, где токи растекаются сверху вниз, поэтому ГЭЦ зависит от климатического состояния Земли. Кроме того, ГЭЦ подвержена влиянию таких факторов космического окружения, как галактические космические лучи и солнечная активность. Так же на ГЭЦ оказывают значительное влияние аэорозоли. Механизмы воздействия данных факторов на ГЭЦ до конца не ясны.%, объяснение механизмов воздействия климатической изменчивости и  на ГЭЦ является актуальной научной задачей.

Аналитическое нахождение распределения крупномасштабных электрических полей в атмосфере в общем случае не возможно, поэтому для исследования ГЭЦ используется численное моделирование. При моделировании ГЭЦ значительные трудности возникают с заданием распределения источников и проводимости воздуха в атмосфере. Отчасти это связано с недостатком наблюдательных данных. Модели ГЭЦ разнятся по используемой геометрии, например, некоторые модели рассматривают атмосферу как сферический слой, а в некоторых атмосфера разбивается на невзаимодействующие столбцы воздуха (так называемая столбцовая модель ГЭЦ).

В первой части дипломной работы реализована столбцовая модель ГЭЦ с учётом параметризации проводимости, предложенной в [?]. Результаты такой модели сравнивались с результатами уже зарекомендовавшей себя модели, разработанной в \cite{Ilin_et_al_2020}, где используется более простое задание проводимости. На основе сопоставления моделей с двумя параметризациями проводимости оценена надобность использования более сложной параметризации проводимости при моделировании ГЭЦ.

Во второй части дипломной работы исследовалась связь колебания Маддена--Джулиана (КМД) с ГЭЦ. КМД является доминирующей компонентой климатической изменчивости в тропиках на временных масштабах в десятки дней. КМД происходит нерегулярно и обычно длится 30--90 дней. Важным аспектом КМД является связанность процессов крупномасштабной атмосферной циркуляции и процессов глубокой конвекции; в течение каждого цикла КМД крупномасштабная связанная структура переносится на восток со скоростью $5\,\textnormal{м с}^{-1}$. Данный эффект затрагивает все долготы, но наибольшее проявление имеет над Восточным полушарием. За последние 50 лет КМД было широко изучено с климатологической точки зрения \cite{Madden_Julian_1994, Zhang_2005, Zhang_et_al_2020}; было установлено, что КМД воздействует на глобальное распределение дождей, на развитие тропических циклонов и даже на Эль-Ниньо/Южное колебание (ЭНЮК).

Однако, лишь несколько исследований было посвящено влиянию КМД на атмосферное электричество. В работе \cite{Anyamba_et_al_2000} показано на основе анализа спутниковых данных и измерений резонансов Шумана в Антарктике, что внутри-сезонная вариация глубокой конвекции отражается в вариации интенсивности шумановских резонансов. Резонансы Шумана возбуждаются молниевыми разрядами облако-земля, поэтому не удивительно, что изменение в глубокой конвекции (которая часто связана с молниевой активностью) отражается на их интенсивности. Ещё одно исследование по данной тематике \cite{Beggan_Musur_2019} показывает, что интенсивность и частота резонансов Шумана коррелирует с индексами, описывающими КМД, но только в течение холодной фазы ЭНЮК.

Молниевая активность (а следовательно и шумановские резонансы) связаны с глубокой конвекцией лишь косвенно. Гораздо более натуральный подход заключается в рассмотрении ГЭЦ, источниками для которой служат квазистационарные токи разделения зарядов как в грозовых облаках, так и в ESC (electrified shower clouds), в которых нет молний; такие токи непосредственно связаны с глубокой конвекцией.

В недавних работах \cite{Slyunyaev_et_al_2021a,Slyunyaev_et_al_2021b} на основе моделирования ГЭЦ было показано, что изменения в глубокой конвекции в течение ЭНЮК модулирует ИП и его суточную вариацию. Результаты данных исследований нашли подтверждение в экспериментальных измерениях ГП \cite{Harrison_et_al_2011,Slyunyaev_et_al_2021c}. Похожий метод был применён в настоящей работе при исследовании связи ГЭЦ с КМД с использованием как результатов численного моделирования ГЭЦ, так и результатов измерений электрического поля в Антарктиде.
% Что было сделано в этой работе:
% 1) написана столбцовая модель ГЭЦ, обнаружено, что учёт более точной параметризации проводимости никак не сказывается на моделировании ИП по сравнению с моделированием для экспаненциальной проводимости
% 2) КМД и ГЭЦ