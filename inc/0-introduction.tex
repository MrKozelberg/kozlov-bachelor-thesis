\section*{ВВЕДЕНИЕ}
\addcontentsline{toc}{section}{ВВЕДЕНИЕ}

В земной атмосфере протекают процессы, формирующие климат Земли, что делает изучение атмосферы критически важным для человека. Атмосферное электричество относится к числу наиболее актуальных направлений в науке, изучающей физику атмосферы. Одна из главных задач атмосферного электричества --- разработка согласующейся с экспериментальными результатами теоретической модели, описывающей распределение электрических полей в атмосфере.

Ключевым понятием атмосферного электричества является глобальная электрическая цепь (ГЭЦ), в которой выделяют две составляющие: ГЭЦ постоянного и переменного тока. Под ГЭЦ постоянного тока обычно понимают квазистационарные токи в атмосфере, поддерживаемые процессами зарядки в электрически активных облаках \cite{Williams_Mareev_2014, Rycroft_et_al_2008, Williams_2009}; под ГЭЦ переменного тока понимают резонансы Шумана, возбуждаемые электрическими разрядами в резонаторе Земля --- ионосфера. Всюду ниже будет рассматриваться ГЭЦ постоянного тока.

Интенсивность ГЭЦ естественно характеризовать ионосферным потенциалом (ИП), который определяется как разность потенциалов между поверхностью Земли и поверхностью, расположенной в ионосфере. Замечательной особенностью ИП является то, что он в первом приближении не зависит от географического места измерения. Экспериментально ИП определяется на основе измерений высотного профиля вертикальной компоненты электрического поля. Интегрированием такого профиля получают значение ИП. Однако измерений ИП очень мало и они не дают полноценной картины \cite{Markson_2007}. Более широко измеряется приповерхностный градиент потенциала (ГП) электрического поля Земли. Недостатком ГП является его подверженность влиянию локальных эффектов. Уменьшить искажения, вносимые локальными эффектами в результаты измерений ГП, удается с помощью усреднения данных измерений по достаточно долгому периоду времени. Усредненные значения ГП прямо пропорциональны ИП.

ГЭЦ объединяет в себе области источников (коими выступают грозовые облака и облака с развитой электрической структурой, но без гроз; такие облака принято называть в англоязычной литературе electrified shower clouds), где в среднем электрические токи поднимаются вверх от поверхности Земли к ионосфере, и области хорошей погоды, где токи текут сверху вниз. Так как источниками в ГЭЦ выступают конвективные облака, ГЭЦ зависит от климатического состояния Земли. Кроме того, ГЭЦ подвержена влиянию со стороны космического окружения Земли: галактические космические лучи и солнечная активность влияют на процессы ионизации в атмосфере. Также на ГЭЦ оказывают значительное влияние аэрозоли.

Аналитическое нахождение распределения крупномасштабных электрических полей в атмосфере в общем случае не возможно, поэтому для исследования ГЭЦ используется численное моделирование. При моделировании ГЭЦ значительные трудности возникают с заданием распределения источников, так как теоретический аппарат, описывающий формирование облака с развитой электрической структурой, не разработан до конца. Модели ГЭЦ разнятся по используемой геометрии, например, некоторые модели рассматривают атмосферу как сферический слой, а в некоторых атмосфера разбивается на столбцы воздуха.

В первой части дипломной работы программно реализована столбчатая модель ГЭЦ с учетом параметризации источников, обсуждаемой в \cite{Ilin_et_al_2020}. С помощью данной модели оказалось возможным сравнить результаты моделирования ГЭЦ с учетом достаточно грубой параметризации проводимости (когда проводимость зависит лишь от высоты) и результаты моделирования с учетом более реалистичной параметризации проводимости, описанной в \cite{Slyunyaev_et_al_2015}. Такое сравнение позволило оценить влияние учета более точной параметризации проводимости атмосферы на моделируемые значения ИП.

%Конвективной деятельностью называют любые проявления конвекции в атмосфере: развитие восходящих и нисходящих токов воздуха, облаков и осадков конвекции, гроз, шквалов, смерчей и тромбов, тайфунов или ураганов и т. д. В метеорологии конвекцию разделяют на мелкую и глубокую [2]. Основное отличие глубокой конвекции от мелкой состоит в том, что она развивается в атмосферном слое большой мощности и важную роль в ее развитии играют процессы, связанные с фазовыми переходами влаги в атмосфере. Другая особенность глубокой конвекции состоит в том, что вследствие больших вертикальных и горизонтальных масштабов существенно возрастает влияние горизонтальной неоднородности метеорологических полей синоптического масштаба, эффекта вращения Земли и неоднородности подстилающей поверхности [8].

Во второй части дипломной работы исследовалась связь колебания Маддена--Джу\-ли\-ана (КМД) с ГЭЦ. КМД является доминирующей компонентой климатической изменчивости в тропиках на временных масштабах в десятки дней. КМД происходит нерегулярно и имеет характерный временной масштаб в 30--90 дней. Важным аспектом КМД является связанность процессов крупномасштабной атмосферной циркуляции и процессов глубокой конвекции; в течение каждого цикла КМД крупномасштабная связанная структура переносится на восток со скоростью около $5\,\textnormal{м}/ \textnormal{с}$. Стоит отметить, что к процессам глубокой конвекции относится любая конвективная деятельность, происходящая на достаточно больших вертикальных и горизонтальных масштабах и сопровождающаяся процессами, связанными с фазовыми переходами влаги в атмосфере. Эффект переноса конвективной структуры на восток затрагивает все долготы, но наиболее значительное проявление имеет над Восточным полушарием.

За последние 50 лет КМД было широко изучено с климатологической и метеорологической точек зрения \cite{Madden_Julian_1994, Zhang_2005, Zhang_et_al_2020}; было установлено, что КМД воздействует на глобальное распределение дождей, на развитие тропических циклонов и даже на Эль-Ниньо/Южное колебание (ЭНЮК), одну из важнейших глобальных климатических мод. Однако лишь несколько исследований было посвящено связям КМД с атмосферным электричеством. В работе \cite{Anyamba_et_al_2000} показано на основе анализа спутниковых данных и измерений резонансов Шумана в Антарктике, что вариация глубокой конвекции, происходящая на временном масштабе 20--30 дней, отражается в вариации интенсивности шумановских резонансов. Резонансы Шумана возбуждаются молниевыми разрядами облако --- земля, поэтому не удивительно, что изменение в глубокой конвекции (которая часто связана с молниевой активностью) отражается на их интенсивности. Еще одно исследование по данной тематике \cite{Beggan_Musur_2019} показывает, что интенсивность и частота резонансов Шумана коррелирует с индексами, описывающими КМД, но только в течение холодной фазы ЭНЮК.

Молниевая активность (а следовательно и шумановские резонансы) связаны с глубокой конвекцией лишь косвенно. Гораздо более естественный подход заключается в рассмотрении ГЭЦ, источниками для которой служат квазистационарные токи разделения зарядов как в грозовых облаках, так и в негрозовых облаках с развитой электрической структурой; такие токи непосредственно связаны с глубокой конвекцией.

В недавних работах \cite{Slyunyaev_et_al_2021a,Slyunyaev_et_al_2021b} на основе моделирования ГЭЦ было показано, что изменения в глубокой конвекции в течение ЭНЮК модулируют ИП и его суточную вариацию. Результаты данных исследований согласуются с экспериментальными измерениями ГП \cite{Harrison_et_al_2011,Slyunyaev_et_al_2021c}. Похожий метод был применен в настоящей работе при исследовании связи ГЭЦ с КМД с использованием как результатов численного моделирования ГЭЦ, так и результатов измерений электрического поля в Антарктиде.
% Что было сделано в этой работе:
% 1) написана столбцовая модель ГЭЦ, обнаружено, что учет более точной параметризации проводимости никак не сказывается на моделировании ИП по сравнению с моделированием для экспаненциальной проводимости
% 2) КМД и ГЭЦ