\Subsection{ЭФФЕКТЫ КМД В РЕЗУЛЬТАТАХ ИЗМЕРЕНИЙ ЭЛЕКТРИЧЕСКОГО ПОЛЯ}

Было показано, что моделируемый ИП имеет синусоидальную вариацию по фазам КМД. Теперь следует исследовать на наличие подобного эффекта результаты измерений ГП на антарктической станции Восток, которые проводились в 2006--2020 (см. раздел \ref{sec:vostok}).

Средние значения ГП, измеренного в хорошую погоду на станции Восток, отвечающие различным фазам КМД продемонстрированы на рис. \ref{fig:pg_vs_mjo_phase}. Снова заметна синусоидальная кривая, но менее гладкая, чем та, что получалась для ИП (см. рис. \ref{fig:ip_vs_mjo_phase}); вариация ГП имеет максимум в фазе~2 и минимум около фаз 5--7. Точнее, вариация ГП имеет 2 локальных минимума --- один в фазе~5 и один в фазе~7, которые разделены малым локальным максимумом в фазе 6; однако, так как значения ГП в фазах 5--7 близки, то далее минимумы в фазе~5 и фазе~7 будут пониматься как один минимум.

Если сравнивать динамику моделируемого ИП за 1980--2020 (см. рис. \ref{fig:ip_vs_mjo_phase}) и измеренного в хорошую погоду на станции Восток ГП за 2006--2020 (см. рис. \ref{fig:pg_vs_mjo_phase}), то можно заметить разницу между двумя такими вариациями. Чтобы сравнение ИП и ГП стало корректным, следует рассматривать значения ГП и значения ИП, усреднённые за одинаковый временной период 2006--2020, что сделано на рис. \ref{fig:ip_pg_vs_mjo_phase}. Коэффициент корреляции между двумя вариациями составляет $r=0.50$ (чего не достаточно для статистической значимости на уровне 1\%), но между вариациями присутствует фазовый сдвиг на восьмую часть периода (т.е. на 1 фазу КМД). Если сдвинуть вариация ГП на 1 фазу вправо, то коэффициент корреляции возрастёт и составит $r=0.90$, чего хватает для статистической значимости на уровне 1\%.

Кроме того, интересно сравнить ГП за дни хорошей погоды с компонентами индекса RMM. Коэффициент корреляции между ГП и RMM1 равен $r=-0.68$, а между ГП и RMM2 коэффициент корреляции составляет $r=-0.67$; ГП одинаковым образом негативно коррелирует с RMM1 и RMM2. Для статистической значимости на уровне 1\% (для чего требуется $\abs{r}\ge0.83$) коэффициенты корреляции слишком малы.

Если рассматривать линейную комбинацию компонент индекса RMM (\ref{rmm_direction}), то максимум корреляции между такой линейной комбинацией и ГП, измеренного в дни хорошей погоды, имеет максимальное значение $r=0.95$ и достигается при $\phi=225^\circ$ (см. рис. \ref{fig:r}); такое направление отмечено на рис. \ref{fig:rmm_diagram} оранжевой штрихованной линией. Следует заметить, что такое направление совпадает с биссектрисой третьей четверти фазовой плоскости индекса RMM, что хорошо соотносится с ранними замечаниями о том, что ГП примерно одинаково негативно коррелирует с RMM1 и RMM2. Кроме того, следует отметить, что оптимальное направление для ГП $\phi=225^\circ$ отличается от найденного для моделируемого ИП ($\phi=90^\circ$) на 65\textdegree, что соответствует примерно 1.5 фазам КМД.

Таким образом, было установлено, что как значения ИП, так и значения ГП (усреднённые по фазам КМД) коррелируют с циклом КМД на статистически значимом уровне в 1\%, но их вариации по фазам КМД имеют фазовый сдвиг друг относительно друга. Штрихи на рис. \ref{fig:variations} обозначают плюс и минус одну стандартную ошибку (что равно стандартному отклонению среднего). Из рис. \ref{fig:ip_pg_vs_mjo_phase} видно, что несогласие между формами двух вариаций не могут быть объяснены только лишь статистическими ошибками. Другие возможные объяснения различия вариаций заключаются в неточности используемой параметризации ИП, в воздействии локальных эффектов на результаты измерений ГП и в воздействии солнечной активности. Эти вопросы будут обсуждены ниже в разделе ? .