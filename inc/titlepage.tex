\begin{titlepage}
	\begin{center}
		{\textbf{\scriptsizeМИНИСТЕРСТВО НАУКИ И ВЫСШЕГО ОБРАЗОВАНИЯ РОССИЙСКОЙ ФЕДЕРАЦИИ}\\
		\textbf{\smallФедеральное государственное автономное образовательное учреждение высшего}\\
		\textbf{\smallобразования «Национальный исследовательский Нижегородский}\\
		\textbf{\smallгосударственный университет им. Н.И. Лобачевского» (ННГУ)}}\\
		\vspace{0.2cm}
		\large{Высшая школа общей и прикладной физики}\\
		\vspace{2cm}
		\Large{\textbf{ГЛОБАЛЬНАЯ АТМОСФЕРНАЯ ЭЛЕКТРИЧЕСКАЯ ЦЕПЬ И КОЛЕБАНИЕ МАДДЕНА--ДЖУЛИАНА}}
	\end{center}
	\vfill
	\begin{singlespacing}
	\begin{tabular}{ll}
		\hspace{8cm} & \begin{tabular}[c]{@{}l@{}} Выпускная квалификационная работа\\ студента 4 курса по направлению\\ подготовки 03.03.02 Физика,\\ профиль – фундаментальная физика,\\ Козлова Александра Владимировича\end{tabular}\\
		& \\
		& \\
		\hspace{8cm} & \begin{tabular}[c]{@{}l@{}}\underline{Научный руководитель}:\\ научный сотрудник ИПФ РАН,\\ кандидат физико-математических наук\\ \\ \underline{\hspace{3.5cm}}Н.Н.~Слюняев\end{tabular}\\
		& \\
		& \\
		\hspace{8cm} & \begin{tabular}[c]{@{}l@{}}\underline{Рецензент}:\\ старший научный сотрудник ИПФ РАН,\\ кандидат физико-математических наук\\ \\ \underline{\hspace{3.5cm}}Е.М.~Лоскутов\end{tabular}\\
		& \\
		& \\
		\hspace{8cm} & \begin{tabular}[c]{@{}l@{}}\underline{Декан~ВШОПФ}:\\ кандидат физико-математических наук\\ \\ \underline{\hspace{3.5cm}}E.Д.~Господчиков\end{tabular}\\
	\end{tabular}
	\end{singlespacing}
	\vfill
	\begin{center}
		Нижний Новгород\\
		2022 г.
	\end{center}
	
\end{titlepage}