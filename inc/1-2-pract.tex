\subsection{РЕАЛИЗАЦИЯ СТОЛБЧАТОЙ МОДЕЛИ}

В рамках столбчатой модели получается достаточно простое выражение для ИП \eqref{eq:ip_stolb}, однако возникают трудности при задании источников и проводимости, виной чему малое количество наблюдательных данных и их противоречивость. Ниже будут описаны параметризации источников и проводимости атмосферы, которые были использованы в данной работе.

\subsubsection{ПАРАМЕТРИЗАЦИЯ ИСТОЧНИКОВ ГЭЦ}
\label{sec:sources}

В данной работе будет использована параметризация плотности тока источников, которая подробно обсуждалась в \cite{Ilin_et_al_2020}. Следует заметить, что рассматриваемая параметризация строится на результатах моделирования атмосферной динамики с помощью модели Weather Research and Forecasting (WRF) на широтно-долготной сетке 1\textdegree\texttimes1\textdegree. Ячейки такой сетки служат основаниями для столбцов модели ГЭЦ. То есть разрешение по широте и долготе, на котором придется в дальнейшем работать, составляет порядка $100\, \textnormal{км}$ вдоль долготы и вдоль широты.

Не каждый столбец содержит в себе источник. Источники есть лишь в столбцах, где присутствуют облака с развитой электрической структурой, для формирования которых нужна достаточно сильная конвекция. Поэтому появляется критерий, согласно которому во всех столбцах с максимальным по высотному профилю значением доступной конвективной потенциальной энергии (в англоязычной литературе часто используют термин convective available potential energy или CAPE) меньше порогового значения $\varepsilon_0 = 1000\, \text{Дж}/ \text{кг}$ нет токовых источников.

Далее следует заметить, что если в столбце и есть облака с развитой электрической структурой, то они не покрывают всей площади основания столбца, так как размер оснований, как отмечалось выше, крайне велик. Поэтому вводят некий коэффициент, который отражает то, какая часть площади основания столбца покрыта облаками, вносящими вклад в ГЭЦ. Таким коэффициентом выступает отношение $\alpha_i \propto P_i/W_i$ \cite{Mareev_Volodin_2014}, где $W_i$ --- количество воды, которое может выпасть в виде осадков в данном столбце, а $P_i$ --- количество выпавших осадков, посчитанное за симметричный двух-часовой интервал. Коэффициент пропорциональности обычно не важен и переносится в константу $j_0$ (см. формулу \eqref{eq:ip}).То есть площадь основания столбца разбивается на площадь области, свободной от источников, и на площадь области, ими занятую:
\begin{equation}
	S_i = S^\text{без ист.}_i + S^\text{ист.}_i,\; S^\text{ист.}_i = \alpha_i S_i.
\end{equation}
Только в части столбца над площадью $S^\text{ист.}_i$ ток источников отличен от нуля.

Осталось определить высотный профиль плотности токов источников, который для простоты задается П-образной функцией
\begin{equation}
	(j_s)_i(z) = 
	\begin{cases}
		j_0, &\text{ если }z\in\qty(z^1_i,z^2_i);\\
		0, &\text{ если }z\ \overline{\in}\ \qty(z^1_i,z^2_i);
	\end{cases}
\end{equation}
где $j_0$ --- характерная величина тока разделения зарядов в облаках с развитой электрической структурой, а через $z^1_i$ и $z^2_i$ обозначены высоты изотерм, соответствующих $0^\circ\text{C}$ и $-38^\circ\text{C}$. Такие высоты показывают характерные границы области смешанной фазы.

\subsubsection{ПРОВОДИМОСТЬ ВОЗДУХА, ЗАВИСЯЩАЯ ЛИШЬ ОТ ВЫСОТЫ}
\label{sec:exp_sigma}

Зачастую в качестве проводимости воздуха используют крайне простую функцию, которая зависит лишь от высоты, и имеет вид:
\begin{equation}
    \sigma(z) = \sigma_0 \exp\qty(z/H),
    \label{eq:cond}
\end{equation}
где $\sigma_0$ --- приповерхностное значение проводимости воздуха, а через $H$ обозначен характерный масштаб увеличения проводимости; обычно полагают $H = 6\, \textnormal{км}$. Тогда, подставляя параметризацию источников, описанную в разделе \ref{sec:sources}, и функцию проводимости \eqref{eq:cond} в выражения для ИП \eqref{eq:ip_stolb}, можно прийти к следующей формуле для ИП:
\begin{equation}\label{eq:ip}
 	V = \dfrac{j_0H}{\sigma_0 S_E} \sum\limits_i \qty[\exp\qty(-\dfrac{z^1_i}{H}) - \exp\qty(-\dfrac{z^2_i}{H})] \times \dfrac{P_i S_i}{W_i} \times \theta(\varepsilon_i - \varepsilon_0),
\end{equation}
где $S_E$ --- площадь поверхности Земли,  $\varepsilon_i$ --- максимальное значение CAPE в столбце, а под функцией $\theta(x)$ понимается функция Хевисайда. Сумма (\ref{eq:ip}) берется по всем столбцам модели. Слагаемые данной суммы принято называть вкладами в ИП. Более глубокий анализ данной формулы производится в \cite{Ilin_et_al_2020}. Важно повторить основные идеи данной параметризации: второй множитель в формуле (\ref{eq:ip}) оценивает площадь, занимаемую облаками в каждой из ячеек модели, а третий множитель позволяет выделить столбцы с развитой конвективной активностью по критерию, основанному на значении CAPE. Для расчета ИП необходимо задать параметры (высоты изотерм, осадки и CAPE), которые берутся из результатов воспроизведения атмосферной динамики.

\subsubsection{ПРОВОДИМОСТЬ ВОЗДУХА, ЗАВИСЯЩАЯ ОТ СОЛНЕЧНОЙ АКТИВНОСТИ}
\label{sec:complicated_sigma}

В данном разделе будет рассмотрена более сложная параметризация проводимости, основанная на ряде экспериментальных измерений и подробно обсуждаемая в \cite{Slyunyaev_et_al_2015}. Главной особенностью данной параметризации проводимости воздуха является учет зависимости проводимости от солнечной активности, которая модулирует поток галактических космических лучей, сильно влияющих на процессы ионизации и рекомбинации в атмосфере. Такую параметризацию проводимости не удается записать компактно, поэтому запись данной проводимости с помощью формул будет опущена в настоящей работе. В рамках данной параметризации проводимость зависит от высоты $z$, геомагнитной широты $\lambda$ и параметра $\xi\in[0,\, 1]$, характеризующего уровень солнечной активности. $\xi=0$ соответствует минимуму солнечной активности, $\xi=1$ --- максимуму. Геомагнитные координаты зависят от положений магнитных полюсов, положение которых со временем меняется; кроме того, со временем меняется и уровень солнечной активности, поэтому проводимость в рамках данной параметризации зависит и от времени, хоть и неявно.

Если рассматривать параметризацию источников, описанную в разделе \ref{sec:sources}, то выражение для ИП \eqref{eq:ip_stolb} примет вид
\begin{equation}
    V = \sum\limits^N_{i=1}j_0\qty(\dfrac{P_i S_i}{W_i} \times \theta(\varepsilon_i - \varepsilon_0) \times \int\limits_{z_i^1}^{z_i^2} \dfrac{
    \dd{z}}{\sigma_i} \Bigg/
    \int\limits_0^{H_0} \dfrac{\dd{z}}{\sigma_i})
    \Bigg/
    \sum\limits^N_{i=1}\qty( S_i \Bigg/ {\int\limits_0^{H_0} \dfrac{\dd{z}}{\sigma_i}}).
    \label{eq:ip_stolb_1}
\end{equation}
Таким образом, сосчитать ИП в рамках столбчатой модели ГЭЦ с реалистичной проводимостью, учитывающей влияние солнечной активности, возможно лишь путем численного интегрирования обратной проводимости; кроме того, как и в случае с экспоненциальной проводимостью, для расчета ИП нужны результаты воспроизведения атмосферной динамики (CAPE, данные по осадкам, запасенной влаге и высотам изотерм).





% возьми ход мысли из статьи Slyunayev_et_al_2019, оттуда описание моделирования WRF
% передай логику моделирования, сначала моделирования атмосферной динамики, откуда берутся источники, далее откуда была взята проводимость (1 абзац про проводимость), потом это все подставляется в формулу и численно интегрируется, и получается ИП и высотный профиль электрического потенциала

% следующий подраздел будет про сравнение моделей с двумя разными проводимостями, перенеси описание той модели сюда, только аккуратно


