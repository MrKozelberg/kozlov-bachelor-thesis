\Subsection{ИЗМЕРЕНИЯ ЭЛЕКТРИЧЕСКОГО ПОЛЯ НА СТАНЦИИ ВОСТОК}

За исключением моделирования ГЭЦ с моделью WRF, в настоящей части работы используются результаты измерений ГП на российской антарктической станции Восток (78.5\textdegree\ ю. ш., 106.9\textdegree\ в. д., $3488\,\textnormal{м}$ над уровнем моря) за период 2006--2020. Собираемые в чистой локации на высоте $3488\,\textnormal{м}$ над уровнем моря такие измерения ГП представляют уникальный длинный набор данных, описывающих ГЭЦ \cite{Burns_et_al_2012,Burns_et_al_2017}.

Электрическое поле измеряется с помощью вращающегося диполя, данный датчик был установлен на станции Восток в конце 2005 года в рамках российской--австралийского соглашения \cite{Burns_et_al_2017}. Вращающийся диполь установлен на высоте $3\,\textnormal{м}$ на уровнем снежного покрова, возвышаясь над основными строениями станции. Величины ГП собираются в форме средних за 10-секундные интервалы значений. Данные имеют большой пропуск во второй половине 2017 года и несколько пропусков поменьше в иное время, все пропуски имеют технические причины. Для некоторых периодов времени доступны только 5-минутные данные.

Чтобы упростить анализ, данные измерений усреднялись по часам UTC, при этом, если как 10-секундные данные, так и 5-минутные данные доступны, брались 10-секундные. При усреднении рассматривались лишь те часы, для которых хотя бы 80\% 10-секундных или 5-минутных значений ГП было измерено. Все измеренные значения ГП делились на форм-фактор, равный $3$, для устранения помех, вызванных металлическим стержнем, поддерживающим датчик электрического поля.

Дни хорошей погоды выбирались на основе подхода, применённого в \cite{Slyunyaev_et_al_2021a}. Чтобы выделить дни хорошей погоды, не обязательно использовать метеорологические данные, можно воспользоваться критерием, основанным на значениях ГП, который обычно работает достаточно хорошо \cite{Burns_et_al_2012,Burns_et_al_2017}. Следующая формальная процедура применялась к наборам данных ГП:
\begin{enumerate}
	\item Исключаются дни с неполными или пропущенными часовыми значениями.
	\item Из рассмотрения убираются дни с отрицательными или нулевыми значениями ГП.
	\item Исключаются дни с часовыми значениями ГП, превышающими $300\,\textnormal{В}/\textnormal{м}$.
	\item Среди оставшихся дней удерживаются только те, в которых разница между максимумом суточной вариации и её минимумом не превышает 150\% от среднего за день значения.
\end{enumerate}
