\Anonsection{ВВЕДЕНИЕ}

В земной атмосфере протекают процессы, формирующие климат Земли, что делает изучение атмосферы критически важным для человека. Одним из наиболее актуальных направлений фундаментальных исследований атмосферы является атмосферное электричество. К задачам данного направления относятся изучение электрического окружения Земли и установление связей электрических параметров атмосферы с климатической изменчивостью. Ответы на эти вопросы позволят учесть при климатическом моделировании взаимодействие климатической системы с электрическим окружением Земли, что может привести к повышению точности и устойчивости климатических прогнозов.

Одним из ключевых понятий атмосферного электричества является глобальная электрическая цепь (ГЭЦ) \cite{Williams_Mareev_2014}. ГЭЦ представляет собой распределённый токовый контур, образованный слоем воздуха между землёй и ионосферой. Выделяют два типа ГЭЦ: переменного тока и постоянного. В ГЭЦ первого типа источниками выступают молниевые разряды облако-земля, в ГЭЦ постоянного тока источниками являются токи разделения зарядов в облаках с развитой электрической структурой. Всюду ниже будет рассматриваться ГЭЦ постоянного тока.

Интенсивность ГЭЦ характеризуется ионосферным потенциалом (ИП), который определяется как разность потенциалов на ионосфере и на земле. Замечательной особенностью ИП является то, что он в первом приближении не зависит от географического места измерения.

ГЭЦ объединяет в себе области плохой погоды, где в среднем электрические токи поднимаются вверх от поверхности земли к ионосфере, и области хорошей погоды, где токи растекаются сверху вниз, поэтому ГЭЦ зависит от климатического состояния Земли. Кроме того, ГЭЦ подвержена влиянию таких факторов космического окружения, как галактические космические лучи и солнечная активность. Механизмы воздействия данных факторов на ГЭЦ до конца не ясны, объяснение механизмов взаимодействия ГЭЦ с климатом и с космическими лучами является актуальной научной задачей.

Аналитическое нахождение распределения крупномасштабных электрических полей в атмосфере в общем случае не возможно, поэтому для исследования ГЭЦ используется численное моделирование. Одной из задач данной работы являлась реализация столбцовой модели ГЭЦ, в которой была использована параметризация проводимости, учитывающая 11 летний солнечный цикл... и её последующие сравнение с моделью \cite{Ilin_et_al_2020}, где использовалась более простое задание проводимости.

Связь ГЭЦ с климатом прослеживается по наличию устойчивых вариаций ИП, совпадающих по масштабу с отвечающей им климатической изменчивостью. Например, давно известны вариации ИП на временных масштабах, связанных с сезонным и суточным циклами \cite{Lavigne_et_al_2017}. Продолжается и открытие новых связей ГЭЦ с климатом. В недавних исследованиях \cite{Slyunyaev_et_al_2021a,Slyunyaev_et_al_2021b,Slyunyaev_et_al_2021c} была обнаружена связь вариации ИП с Эль-Ниньо/Южное колебание (ЭНЮК), которое является второй по значимости климатической модой Земли после сезонного цикла и определяет климат тропической части Тихого океана на масштабах в несколько месяцев.

В качестве второй задачи данной работы было выбрано изучение ГЭЦ на наличие паттернов, связанных с другой климатической модой, отвечающей за вариацию климата в тропиках,~---~колебанием Маддена--Джулиана (КМД), которое происходит на временных масштабах в несколько десятков дней над приэкваториальной территорией Земли \cite{Zhang_2005}. Методика таких исследований была основана, как и в \cite{Slyunyaev_et_al_2021a,Slyunyaev_et_al_2021b,Slyunyaev_et_al_2021c}, на анализе результатов моделирования ГЭЦ, а также на анализе временных рядов климатологического индекса, характеризующего КМД.

% Что было сделано в этой работе:
% 1) написана столбцовая модель ГЭЦ, обнаружено, что учёт более точной параметризации проводимости никак не сказывается на моделировании ИП по сравнению с моделированием для экспаненциальной проводимости
% 2) КМД и ГЭЦ