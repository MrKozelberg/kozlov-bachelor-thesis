\Anonsection{ВВЕДЕНИЕ}

В земной атмосфере протекают процессы, формирующие климат Земли, что делает изучение атмосферы критически важным для человека. Одним из наиболее актуальных направлений фундаментальных исследований атмосферы является атмосферное электричество. К задачам данного направления относятся изучение электрического окружения Земли и установление связей электрических параметров атмосферы с климатической изменчивостью. Ответы на эти вопросы позволят учесть при климатическом моделировании взаимодействие климатической системы с электрическим окружением Земли, что может привести к повышению точности и устойчивости климатических прогнозов.

Одним из ключевых понятий атмосферного электричества является глобальная электрическая цепь (ГЭЦ) \cite{Williams_Mareev_2014}. ГЭЦ представляет собой распределённый токовый контур, образованный слоем воздуха между землёй и ионосферой. Выделяют два типа ГЭЦ: переменного тока и постоянного. В ГЭЦ первого типа источниками выступают молниевые разряды облако-земля, в ГЭЦ постоянного тока источниками являются токи разделения зарядов в облаках с развитой электрической структурой. Всюду ниже будет рассматриваться ГЭЦ постоянного тока.

Интенсивность ГЭЦ характеризуется ионосферным потенциалом (ИП), который определяется как разность потенциалов на ионосфере и на земле. ИП в первом приближении не зависит от точки измерения.


% Что было сделано в этой работе:
% 1) написана столбцовая модель ГЭЦ, обнаружено, что учёт более точной параметризации проводимости никак не сказывается на моделировании ИП по сравнению с моделированием для экспаненциальной проводимости
% 2) КМД и ГЭЦ
\underline{\hspace{\textwidth}}



Атмосфера в значительной степени определяет жизнь человека, она обеспечивает условия необходимые для поддержания жизни и отвечает за формирование климата, что делает изучение атмосферы критически важным для человека. Одним из наиболее актуальных направлений фундаментальных исследований атмосферы является атмосферное электричество. С развитием вычислительной техники и с накоплением большого объёма измерительных данных вырос интерес научного сообщества к данному направлению исследований.

Одним из ключевых понятий атмосферного электричества является глобальная электрическая цепь (ГЭЦ) \cite{Williams_Mareev_2014}. ГЭЦ представляет собой распределённый токовый контур, образованный слоем воздуха, который располагается между двумя поверхностями: поверхностью Земли и поверхностью ионосферы, которые полагаются эквипотенциальными. %Поверхность Земли можно считать эквипотенциальной, так как проводимость как поверхности суши, так и поверхности океана на несколько порядков превосходит проводимость воздуха в нижних слоях тропосферы. Ионосфера же полагается эквипотенциальной из-за её большой проводимости, вызванной значительным количеством ионов, образующихся при взаимодействии молекул воздуха с космическими лучами.
Стационарными источниками в таком токовом контуре выступают облака с развитой электрической структурой.%Возможность формирования данных слоёв обеспечивается тем, что заряд, накопившись в области пространства с малой проводимостью (в нижних слоях тропосферы), очень слабо растекается.

Характеристикой интенсивности ГЭЦ постоянного тока является разность потенциалов между поверхностью Земли и ионосферой, которую называют ионосферным потенциалом (ИП).
Замечательной особенностью ИП является то, что он в первом приближении не зависит от географического места измерения, поэтому можно рассматривать его как интегральную характеристику электрического окружения Земли.

%Прямые измерения ИП в последние десятилетия почти не производятся по причине их дороговизны. Однако существует прямая связь между ИП и измеренным у поверхности Земли в хорошую погоду градиентом потенциала (ГП) электрического поля, что позволяет восстановить относительную вариацию ИП по данным натурных измерений. 

Давно известны устойчивые картины вариации ИП на временных масштабах, связанных с сезонным и суточным циклами \cite{Lavigne_et_al_2017}. Не все механизмы формирования характерных вариаций ИП до конца ясны на данный момент. Поэтому актуальной научной задачей является объяснение физических механизмов их образования. Кроме того, продолжается открытие новых связей ИП с климатом. В недавних исследованиях \cite{Slyunyaev_et_al_2021a,Slyunyaev_et_al_2021b,Slyunyaev_et_al_2021c} была обнаружена связь вариации ИП с Южным колебанием (ЮК), которое является второй по значимости климатической модой Земли после сезонного цикла и определяет климат тропической части Тихого океана на масштабах в несколько месяцев.

В связи с этим представляет интерес изучить вариацию ИП на наличие эффектов, связанных с другой климатической модой, отвечающей за вариацию климата в тропиках. В качестве такой моды было выбрано колебание Маддена–Джулиана (КМД), которое происходит на временных масштабах в несколько десятков дней над приэкваториальной территорией Земли \cite{Zhang_2005}.

Методика таких исследований будет основана, как и в \cite{Slyunyaev_et_al_2021a,Slyunyaev_et_al_2021b,Slyunyaev_et_al_2021c}, на анализе результатов моделирования ГЭЦ постоянного тока, а также на анализе временных рядов климатологического индекса, характеризующего КМД. Кроме того, будет предпринята попытка объяснить физический механизм обнаруженной связи, для чего будет привлечен некоторый метод статистического анализа.